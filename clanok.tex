\documentclass[10pt,twoside,slovak,a4paper]{article}
\usepackage{graphicx}   
\usepackage{todonotes}
\usepackage{lipsum}
\usepackage{subcaption}
\usepackage{cite}
\usepackage[slovak]{babel}
\usepackage{float}
\usepackage{amsmath}

\title{Návrh systému odporúčaní hudby}
\author{Oleksii Khmelevskyi}
\date{}
\begin{document}
\maketitle
\section*{Abstrakt}
S vývojom internetovej technológie a digitálnej hudby spravovanie a hľadanie skladieb sa stalo komplikovanejším. Existuje veľa rôznych algoritmov a systémov, ktoré odporúčajú súvisiacu hudbu z týmito, čo je zaujímavé pre používateľa. Hudobné platformy zbierajú údaje a vytvárajú personalizované návrhy pre zlepšenie používateľskej skúsenosti a zvýšenie času ich používania ľuďmi. Existuje 2 z najpopulárnejších algoritmov pre odporúčanie hudby. Toto sú kolaboratívne filtrovanie a modeli založený na podobne vyzerajúcich obsahoch. Algoritmus podobných môže identifikovať ľudia s ich združeniami, zatiaľ kolaboratívne filtrovanie identifikuje potenciálne spojenia medzi inými ľudmi. Tieto systémy vykazujú dobre výsledky, ale sú stále v ranom štádiu vývoja. Taký úvodný ako emocionálne zloženie práce alebo časove aspekty mohli by dokonale doplniť existujúce údaje pre získanie presnejších výsledkov. Umelá inteligencia takže ukazuje dobré výsledky keď príde na odporúčajúce systémy, kvôli optimalizovanej prace s Big Dáta, ktorá je neoddeliteľnou súčasťou prace s odporúčacími systémami, a preto pridanie Deep Learning, ktorý by hľadal vzory vo piesňach a reláciách počúvania, do analýzu údajov môže zlepšiť rezultát. Hudba je univerzálna a subjektívna, môže vyjadrovať emócie a ovplyvniť náladu poslucháča, ale zároveň vkus v hudbe je pri každého jedinečný, a preto nie vždy tradičné metódy môžu vykonávať dobre. Z toho dôvodu, stálo by to za zostavenie podrobných popisov činnosti súčasných algoritmov, výsledkov existujúcich experimentov a možností pridania ďalších vstupov do týchto schém.
\end{document}
